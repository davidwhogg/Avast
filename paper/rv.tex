\documentclass[12pt, letterpaper]{article}

\newcommand{\acronym}[1]{{\small{#1}}}
\newcommand{\project}[1]{\textsl{#1}}
\newcommand{\Avast}{\project{Avast}}

\newcommand{\unit}[1]{\mathrm{#1}}
\newcommand{\m}{\unit{m}}
\newcommand{\s}{\unit{s}}
\newcommand{\mps}{\m\,\s^{-1}}

\newcommand{\dd}{\mathrm{d}}
\newcommand{\given}{\,|\,}
\newcommand{\normal}{{\mathcal{N}}}
\newcommand{\com}{\mathrm{cm}}
\newcommand{\surf}{\mathrm{sf}}

\setlength{\parindent}{1.0\baselineskip}
\linespread{1.09}
\raggedbottom
\sloppy
\sloppypar
\frenchspacing

\begin{document}

\section*{\Avast: A data-driven method for obtaining extremely precise stellar radial velocities}

\noindent
by DWH, others

\paragraph{Abstract:}
In the best radial-velocity experiments, stellar
spectroscopic data are taken with continuously calibrated hardware,
spectra are extracted with very good wavelength solutions, and radial
velocities are determined by cross-correlation with templates.
These approaches have delivered great science---including many
hundreds of planet discoveries---at the meter-per-second level of
precision.
We expect to find far more science---including truly Earth-like
planets around truly Sun-like stars---when these integrated
hardware--software systems deliver centimeter-per-second-ish
precision.
Here we propose a methodological and software component for this new
era of precision: \Avast, a data-driven method for determining
radial velocities from a set of multiple spectra taken with the same
instrument over multiple epochs.
\Avast\ takes a data-driven approach to the radial-velocity
measurement; it delivers only relative velocities, not absolute
velocities, but it does so at nearly the Cram\'er--Rao bound.
Importantly, \Avast\ is capable of locating and modeling changes in
the stellar spectrum that are covariant with velocity.
That is, if some component of a star's apparent velocity shifts are
not due to stellar radial velocity but simply surface convection,
activity, sunspots, or rotation, and these surface effects create
spectral changes that go beyond pure velocity shifts, \Avast\ can find
them and exploit them to improve radial-velocity measurements.
The fundamental technology underlying \Avast\ is probabilistic
(Bayesian) regression, with some hierarchical parameters.

\section{Introduction}

There is a remarkable collection of spectrographs and data-analysis
pipelines being used for exoplanet discovery and characterization.
They routinely produce radial-velocity measurements with a precision,
or scatter about the mean, of about $1\,\mps$.
Why not better?
If you compute the Cram\'er-Rao bound for typical spectra, it is much
more precise than this; that is, we have the photons and the spectral
features to do better.

One thing we have shown (unpublished, except in blog posts) is that
the problem with these spectrographs is \emph{not} in the software
pipelines.
That is, we can show that spectral calibration and extraction is not
what is causing the precision loss relative to the bounds.
This means that either the problem is in the hardware upstream of the
detector readout, or else in the atmosphere, interstellar medium, or
the star itself.
This project was motivated by the question: Can we improve radial
velocity measurements if the radial-velocity noise is being generated
by the star itself?
That is, can we separate star-generated radial-velocity noise from
true radial-velocity variations of the stellar center of mass?
(In what follows, we will try to be clear that we are talking about
the star's center of mass, not the center of mass of the planetary or
stellar system of which the star is a part; that latter center of mass
doesn't accelerate much!)
Some of what we do will also apply to certain kinds of atmospheric or
instrumental effects, as we will note below.

Imagine that we are trying to make precise measurements of a stellar
velocity, by taking a precise spectrum of a stellar surface,
integrated over the stellar disk visible to us during the
spectroscopic exposure.
Because the star is convective at its surface with a finite number (or
size) of convection cells, and because the star is oscillating,
rotating, showing star spots, and flaring, the velocity we measure
will not be as cleanly related to the center-of-mass motion of the
star as we would like.
What we really measure is a superposition of the stellar
center-of-mass velocity and a mean surface-velocity anomaly, where the
relevant mean is taken by integrating over the visible-to-us surface
of the star for the duration of the exposure, in the bandpass of the
spectrograph (and so on).

It might seem like this surface-velocity anomaly is an intractably
confusing source of noise.
However, if the anomaly is created by a physical process in or near
the surface of the star, it is probably---only probably, I'm
afraid---also associated with a spectral change that goes beyond pure
Doppler Shift.

\section{Assumptions and method}

The most important assumptions we make in this work are the following:
\begin{itemize}\itemsep=0ex
\item There are $N$ multiple extracted spectra of the same star, taken
  at (fictitious) Solar-System barycentric times (epochs) $t_n$. Each
  of these spectra has been continuum normalized properly, and is
  represented on a grid of $M$ wavelength pixels $\lambda_m$ in the
  (fictitious) Solar-System barycentric rest frame. We assume (perhaps
  naively) that both the times $t_n$ and wavelengths $\lambda_m$ are
  very precisely known.
\item At each pixel $m$ of each spectrum $n$ there is Gaussian noise,
  drawn from a distribution with zero mean and known variance
  $\sigma^2_{nm}$.
\item At each epoch $n$, there is a true radial velocity displacement
  $\Delta v_n$ of the (center of mass of the) star, and also some kind
  of spectral distortion.  That is, there is a Doppler shift and a
  spectral change at every epoch.
\item We expect the spectral distortions to be sparse in
  wavelength-space.  That is, we expect only a few, isolated lines to
  participate in any distortion.
\item In the long run (that is, as the number $N$ of epochs grows
  large), there cannot be any true covariance between the spectral
  distortion and the true velocity of the (center of mass of the)
  star. That is, any observed correlation between the spectral
  distortion and the radial velocity must be caused by some kind of
  effect at the stellar surface. More specifically, we expect the
  center-of-mass velocity to be set by some kind of kinematic or
  dynamical model (such as the two-body problem) and any
  distortions to be set by some kind of stochastic process.
\end{itemize}

Hello World!
\begin{eqnarray}
  y_{nm} &=& \mu(\lambda_m; \Delta_n) + \mathrm{scatter} + \mathrm{noise}
  \\
  \mu(\lambda; \Delta) &=& \mu_0(\lambda)
  - \frac{\lambda}{c}\,\left.\frac{\dd \mu}{\dd\lambda}\right|_{v}\,[\Delta_\com + \Delta_\surf]
  + \left.\frac{\dd \mu}{\dd v_\surf}\right|_{\lambda}\,[\Delta_\surf]
  \\
  \mu_0(\lambda) &=& \sum_{k=0}^{K-1} a_k\,g_k(\lambda)
  \\
  \left.\frac{\dd \mu}{\dd\lambda}\right|_{v} &=& \sum_{k=0}^{K-1} a_k\,\frac{\dd g_k}{\dd\lambda}
  \\
  \left.\frac{\dd \mu}{\dd v_\surf}\right|_{\lambda} &=& \sum_{k=0}^{K-1} b_k\,g_k(\lambda)
  \\
  \Delta_\com &=& h(t; \omega)
  \\
  p(\Delta_\surf \given V_\surf) &=& \normal(\Delta_\surf \given 0, V_\surf)
  \quad ,
\end{eqnarray}
where [something]

\end{document}
